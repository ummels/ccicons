% \iffalse meta-comment
%
% Copyright (C) 2011 by Michael Ummels <michael.ummels@rwth-aachen.de>
%
% This work may be distributed and/or modified under the
% conditions of the LaTeX Project Public License, either version 1.3
% of this license or (at your option) any later version.
% The latest version of this license is in
%   http://www.latex-project.org/lppl.txt
% and version 1.3 or later is part of all distributions of LaTeX
% version 2005/12/01 or later.
%
% This work has the LPPL maintenance status `maintained'.
% 
% The Current Maintainer of this work is Michael Ummels.
%
% This work consists of the files ccicons.dtx, ccicons.ins,
% ccicons.map, ccicons.pfb, ccicons.tfm, ccicons-u.enc and
% the derived files ccicons.sty and uccicons.fd.
%
% The files ccicons.pfb and ccicons.tfm have been
% generated from the file cc-icons-svg.zip available from
% http://mirrors.creativecommons.org/presskit/icons/cc-icons-svg.zip
% and released by Creative Commons (http://creativecommons.org) under a
% Creative Commons Attribution 3.0 Unported license
% (http://creativecommons.org/licenses/by/3.0/).
%
% \fi
%
% \iffalse
%<package>\NeedsTeXFormat{LaTeX2e}[1999/12/01]
%<package>\ProvidesPackage{ccicons}[2011/05/22 v1.2 LaTeX support for Creative Commons icons]
%<fontdef>\ProvidesFile{uccicons.fd}[2011/05/22 v1.2 font definitions for U/ccicons]
%
%<*driver>
\documentclass{ltxdoc}

\usepackage[T1]{fontenc}
\usepackage{charter}
\usepackage{inconsolata}
\usepackage{ccicons}
\usepackage{longtable}

\newcommand{\pkg}[1]{\mbox{#1}}
\newcommand{\url}[1]{\mbox{\texttt{#1}}}

\EnableCrossrefs         
\CodelineIndex
\RecordChanges

\begin{document}
\DocInput{ccicons.dtx}
\end{document}
%</driver>
% \fi
%
% \GetFileInfo{ccicons.sty}
%
% \DoNotIndex{\newcommand}
% 
% \title{The \pkg{ccicons} package}
% \author{Michael Ummels \\ \texttt{michael.ummels@rwth-aachen.de}}
% \date{\fileversion\ -- \filedate}
% \maketitle
%
% \begin{abstract}
% \noindent
% This package offers authors who want to publish their documents under a
% Creative Commons license an easy way to include the relevant icons
% in their documents.
% \end{abstract}
%
% \section{Introduction}
%
% Creative Commons (\url{http://creativecommons.org}) licenses have become
% increasingly popular for authors who want to retain their copyright but give
% other people the possibility to share and build upon their work. For each
% of the licenses that Creative Commons offers, there exists a set of icons
% that can be used to identify the respective license. This package defines
% several commands that allow to typeset these icons. Since the icons are
% stored in a PostScript Type 1 font, they can be scaled freely without
% diminishing their visual appearance.
%
% \section{Usage}
%
% To use this package in your \LaTeX\ document, you need to issue
% \begin{quote}
% |\usepackage{ccicons}|
% \end{quote}
% in the preamble of your document. This makes available the following
% commands to typeset the icons provided by Creative Commons:
%
% \newcommand{\typesetcc}[1]%
%  {#1\,{\large #1}\,{\Large #1}\,{\LARGE #1}\,{\huge #1}\,{\Huge #1}}
% \medskip\noindent
% \begin{longtable}[l]{ll}
%   |\ccLogo| & \typesetcc{\ccLogo} \\
%   |\ccAttribution| & \typesetcc{\ccAttribution} \\
%   |\ccShareAlike| & \typesetcc{\ccShareAlike} \\
%   |\ccNoDerivatives| & \typesetcc{\ccNoDerivatives} \\
%   |\ccNonCommercial| & \typesetcc{\ccNonCommercial} \\
%   |\ccNonCommercialEU| & \typesetcc{\ccNonCommercialEU} \\
%   |\ccNonCommercialJP| & \typesetcc{\ccNonCommercialJP} \\
%   |\ccZero| & \typesetcc{\ccZero} \\
%   |\ccPublicDomain| & \typesetcc{\ccPublicDomain} \\
%   |\ccSampling| & \typesetcc{\ccSampling} \\
%   |\ccShare| & \typesetcc{\ccShare} \\
%   |\ccRemix| & \typesetcc{\ccRemix}
% \end{longtable}
%
% \bigskip\noindent
% Additionally, for each of the common Creative Commons licenses there is a
% command to typeset the CC logo together with the icons applicable for that
% license (see \url{http://creativecommons.org/licenses}):
%
% \medskip\noindent
% \begin{tabular}{ll}
%  |\ccby| & \ccby \\
%  |\ccbysa| & \ccbysa \\
%  |\ccbynd| & \ccbynd \\
%  |\ccbync| & \ccbync \\
%  |\ccbynceu| & \ccbynceu \\
%  |\ccbyncjp| & \ccbyncjp \\
%  |\ccbyncsa| & \ccbyncsa \\
%  |\ccbyncsaeu| & \ccbyncsaeu \\
%  |\ccbyncsajp| & \ccbyncsajp \\
%  |\ccbyncnd| & \ccbyncnd \\
%  |\ccbyncndeu| & \ccbyncndeu \\
%  |\ccbyncndjp| & \ccbyncndjp \\
%  |\cczero| & \cczero \\
%  |\ccpd| & \ccpd
% \end{tabular}
%
% \bigskip\noindent
% Please note that all icons that can be typeset using this package are
% trademarks of Creative Commons and are subject to the Creative Commons
% trademark policy (see \url{http://creativecommons.org/policies}).
%
% \section{Version history}
%
% Version 1.0 (2009/11/29): Initial version \\
% Version 1.1 (2009/12/14): New font with additional glyphs \\
% Version 1.2 (2011/05/22): Optimised some glyphs
%
% \StopEventually{}
%
% \iffalse
%<*package>
% \fi
%
% \section{The main style file}
%
% We provide one internal command to access the characters of the font
% directly.
%    \begin{macrocode}
\newcommand{\ccicons@char}[1]{{\usefont{U}{ccicons}{m}{n}\char#1}}
%    \end{macrocode}
%
% The following commands provide high-level access to to the font. We
% define a command for each character in the font.
%    \begin{macrocode}
\newcommand{\ccLogo}{\ccicons@char{0}}
\newcommand{\ccAttribution}{\ccicons@char{1}}
\newcommand{\ccShareAlike}{\ccicons@char{2}}
\newcommand{\ccNoDerivatives}{\ccicons@char{3}}
\newcommand{\ccNonCommercial}{\ccicons@char{4}}
\newcommand{\ccNonCommercialEU}{\ccicons@char{5}}
\newcommand{\ccNonCommercialJP}{\ccicons@char{6}}
\newcommand{\ccPublicDomain}{\ccicons@char{7}}
\newcommand{\ccZero}{\ccicons@char{8}}
\newcommand{\ccSampling}{\ccicons@char{9}}
\newcommand{\ccShare}{\ccicons@char{10}}
\newcommand{\ccRemix}{\ccicons@char{11}}
%    \end{macrocode}
%
% Finally, for each CC license we define a command that prints the CC logo
% together with the icons applicable for that license.
%    \begin{macrocode}
\newcommand{\ccby}%
  {\mbox{\ccLogo\kern0.1em\ccAttribution}}
\newcommand{\ccbysa}%
  {\mbox{\ccLogo\kern0.1em\ccAttribution\kern0.1em\ccShareAlike}}
\newcommand{\ccbynd}%
  {\mbox{\ccLogo\kern0.1em\ccAttribution\kern0.1em\ccNoDerivatives}}
\newcommand{\ccbync}%
  {\mbox{\ccLogo\kern0.1em\ccAttribution\kern0.1em\ccNonCommercial}}
\newcommand{\ccbynceu}%
  {\mbox{\ccLogo\kern0.1em\ccAttribution\kern0.1em\ccNonCommercialEU}}
\newcommand{\ccbyncjp}%
  {\mbox{\ccLogo\kern0.1em\ccAttribution\kern0.1em\ccNonCommercialJP}}
\newcommand{\ccbyncsa}%
  {\mbox{\ccLogo\kern0.1em\ccAttribution\kern0.1em\ccNonCommercial%
  \kern0.1em\ccShareAlike}}
\newcommand{\ccbyncsaeu}%
  {\mbox{\ccLogo\kern0.1em\ccAttribution\kern0.1em\ccNonCommercialEU%
  \kern0.1em\ccShareAlike}}
\newcommand{\ccbyncsajp}%
  {\mbox{\ccLogo\kern0.1em\ccAttribution\kern0.1em\ccNonCommercialJP%
  \kern0.1em\ccShareAlike}}
\newcommand{\ccbyncnd}%
  {\mbox{\ccLogo\kern0.1em\ccAttribution\kern0.1em\ccNonCommercial%
  \kern0.1em\ccNoDerivatives}}
\newcommand{\ccbyncndeu}%
  {\mbox{\ccLogo\kern0.1em\ccAttribution\kern0.1em\ccNonCommercialEU%
  \kern0.1em\ccNoDerivatives}}
\newcommand{\ccbyncndjp}%
  {\mbox{\ccLogo\kern0.1em\ccAttribution\kern0.1em\ccNonCommercialJP%
  \kern0.1em\ccNoDerivatives}}
\newcommand{\cczero}%
  {\mbox{\ccLogo\kern0.1em\ccZero}}
\newcommand{\ccpd}%
  {\mbox{\ccLogo\kern0.1em\ccPublicDomain}}
%    \end{macrocode}
%
% \iffalse
%</package>
%<*fontdef>
% \fi
%
% \section{The font definitions file}
%
% We just need to declare one family with one shape.
%    \begin{macrocode}
\DeclareFontFamily{U}{ccicons}{}
\DeclareFontShape{U}{ccicons}{m}{n}{
   <-> ccicons
}{}
%    \end{macrocode}
%
% \iffalse
%</fontdef>
% \fi
%
% \CheckSum{160}
%
% \CharacterTable
%  {Upper-case    \A\B\C\D\E\F\G\H\I\J\K\L\M\N\O\P\Q\R\S\T\U\V\W\X\Y\Z
%   Lower-case    \a\b\c\d\e\f\g\h\i\j\k\l\m\n\o\p\q\r\s\t\u\v\w\x\y\z
%   Digits        \0\1\2\3\4\5\6\7\8\9
%   Exclamation   \!     Double quote  \"     Hash (number) \#
%   Dollar        \$     Percent       \%     Ampersand     \&
%   Acute accent  \'     Left paren    \(     Right paren   \)
%   Asterisk      \*     Plus          \+     Comma         \,
%   Minus         \-     Point         \.     Solidus       \/
%   Colon         \:     Semicolon     \;     Less than     \<
%   Equals        \=     Greater than  \>     Question mark \?
%   Commercial at \@     Left bracket  \[     Backslash     \\
%   Right bracket \]     Circumflex    \^     Underscore    \_
%   Grave accent  \`     Left brace    \{     Vertical bar  \|
%   Right brace   \}     Tilde         \~}
%
% \Finale
\endinput
